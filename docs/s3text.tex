\documentclass[]{article}
\usepackage{lmodern}
\usepackage{amssymb,amsmath}
\usepackage{ifxetex,ifluatex}
\usepackage{fixltx2e} % provides \textsubscript
\ifnum 0\ifxetex 1\fi\ifluatex 1\fi=0 % if pdftex
  \usepackage[T1]{fontenc}
  \usepackage[utf8]{inputenc}
\else % if luatex or xelatex
  \ifxetex
    \usepackage{mathspec}
  \else
    \usepackage{fontspec}
  \fi
  \defaultfontfeatures{Ligatures=TeX,Scale=MatchLowercase}
\fi
% use upquote if available, for straight quotes in verbatim environments
\IfFileExists{upquote.sty}{\usepackage{upquote}}{}
% use microtype if available
\IfFileExists{microtype.sty}{%
\usepackage{microtype}
\UseMicrotypeSet[protrusion]{basicmath} % disable protrusion for tt fonts
}{}
\usepackage[margin=1in]{geometry}
\usepackage{hyperref}
\hypersetup{unicode=true,
            pdftitle={Supplementary materials: Orienting the causal relationship between imprecisely measured traits using GWAS summary data},
            pdfborder={0 0 0},
            breaklinks=true}
\urlstyle{same}  % don't use monospace font for urls
\usepackage{graphicx,grffile}
\makeatletter
\def\maxwidth{\ifdim\Gin@nat@width>\linewidth\linewidth\else\Gin@nat@width\fi}
\def\maxheight{\ifdim\Gin@nat@height>\textheight\textheight\else\Gin@nat@height\fi}
\makeatother
% Scale images if necessary, so that they will not overflow the page
% margins by default, and it is still possible to overwrite the defaults
% using explicit options in \includegraphics[width, height, ...]{}
\setkeys{Gin}{width=\maxwidth,height=\maxheight,keepaspectratio}
\IfFileExists{parskip.sty}{%
\usepackage{parskip}
}{% else
\setlength{\parindent}{0pt}
\setlength{\parskip}{6pt plus 2pt minus 1pt}
}
\setlength{\emergencystretch}{3em}  % prevent overfull lines
\providecommand{\tightlist}{%
  \setlength{\itemsep}{0pt}\setlength{\parskip}{0pt}}
\setcounter{secnumdepth}{0}
% Redefines (sub)paragraphs to behave more like sections
\ifx\paragraph\undefined\else
\let\oldparagraph\paragraph
\renewcommand{\paragraph}[1]{\oldparagraph{#1}\mbox{}}
\fi
\ifx\subparagraph\undefined\else
\let\oldsubparagraph\subparagraph
\renewcommand{\subparagraph}[1]{\oldsubparagraph{#1}\mbox{}}
\fi

%%% Use protect on footnotes to avoid problems with footnotes in titles
\let\rmarkdownfootnote\footnote%
\def\footnote{\protect\rmarkdownfootnote}

%%% Change title format to be more compact
\usepackage{titling}

% Create subtitle command for use in maketitle
\newcommand{\subtitle}[1]{
  \posttitle{
    \begin{center}\large#1\end{center}
    }
}

\setlength{\droptitle}{-2em}
  \title{Supplementary materials: Orienting the causal relationship between
imprecisely measured traits using GWAS summary data}
  \pretitle{\vspace{\droptitle}\centering\huge}
  \posttitle{\par}
  \author{}
  \preauthor{}\postauthor{}
  \date{}
  \predate{}\postdate{}


\begin{document}
\maketitle

\subsection{S3 Text. The influence of unmeasured confounding on the
inference of causal
directions}\label{s3-text.-the-influence-of-unmeasured-confounding-on-the-inference-of-causal-directions}

We have assumed no unmeasured confounding in these simulations.
Unmeasured confounding will however have potentially large influences on
mediation-based methods for inferring causal directions, and can also
adversely influence the estimate of the causal direction for the Steiger
test.

\subsubsection{Unmeasured confounding in
mediation}\label{unmeasured-confounding-in-mediation}

Including an unmeasured confounder, \(u\), after ignoring intercept
terms the exposure \(x\) and outcome \(y\) variables can be modelled as

\[
\begin{aligned}
y & = \beta_x x + \beta_{uy} u + \epsilon_x \\
x & = \beta_g g + \beta_{ux} u + \epsilon_g
\end{aligned}
\]

The observational estimate of the causal effect of \(x\) on \(y\),
\(\hat{\beta}_x\) is obtained from

\[
\begin{aligned}
\hat{\beta}_x & = cov(x, y) / var(x) \\
& = \frac{\beta_g^2 \beta_x var(g) + \beta_{ux}^2 \beta_x var(u) + \beta_x var(\epsilon_g)} {\beta_g^2 var(g) + \beta_{ux}^2 var(u) + var(\epsilon_g)}
\end{aligned}
\]

From this it is clear that \(\beta_x\) and \(\hat{\beta}_x\) will differ
when both \(\beta_{uy}\) and \(\beta_{ux}\) are non-zero. Relating to
mediation, where we attempt to test if \(g\) associates with \(y\) after
adjusting \(y\) for \(x\), such that

\[
\hat{y}^* = \hat{\beta}_x x
\]

and

\[
\begin{aligned}
cov(g, y - \hat{y}^*) & = cov(g, \beta_x x + \beta_{uy} u + \epsilon_x - \hat{\beta}_x x) \\
& = cov(g, (\beta_x - \hat{\beta}_x)(\beta_g g + \beta_ux u + \epsilon_x)) \\
& = (\beta_x - \hat{\beta}_x) var(g)
\end{aligned}
\]

should any amount of unmeasured confounding exist, therefore, there will
remain an association between \(g\) and \(y|x\), which will introduce
errors in inferring causal directions.

\subsubsection{Unmeasured confounding in the MR Steiger
test}\label{unmeasured-confounding-in-the-mr-steiger-test}

Similarly, we can investigate the extent to which unmeasured confounding
will lead to the wrong causal direction between \(x\) and \(y\) using
the MR Steiger test, evaluating the liability for the inequality
\(cor(g,x)^2 > cor(g,y)^2\) being incorrect. After some algebra

\[
cor(g,x)^2 = \frac{\beta_g^2}{\beta_g^2var(g) + \beta_{ux}^2 var(u) + var(\epsilon_x)}
\]

and

\[
cor(g,y)^2 = \frac{\beta_x^2 \beta_g^2 var(g)^2} {\hat{\beta}_x^2 \beta_g^2 var(g) + \hat{\beta}_x^2 \beta_{ux}^2 var(u) + \beta_{uy}^2 var(u) + var(\epsilon_y)}
\]

S2 fig shows the relationship between the magnitude of the correlations
between \(x\), \(y\) and the confounder \(u\) for a range of
\(\beta_{xy} = (-2,2)\), \(\beta_{gx} = 0.1\) and a range of confounder
effects. The pattern of results were similar for different values of
\(\beta_{gx}\). We note that in most cases for the parameter values
explored, where the observational absolute \(\hat{R}^2_{xy}\) is less
than 0.2, unmeasured confounding will not incur the wrong causal
direction in the MR Steiger test.


\end{document}
