\documentclass[]{article}
\usepackage{lmodern}
\usepackage{amssymb,amsmath}
\usepackage{ifxetex,ifluatex}
\usepackage{fixltx2e} % provides \textsubscript
\ifnum 0\ifxetex 1\fi\ifluatex 1\fi=0 % if pdftex
  \usepackage[T1]{fontenc}
  \usepackage[utf8]{inputenc}
\else % if luatex or xelatex
  \ifxetex
    \usepackage{mathspec}
  \else
    \usepackage{fontspec}
  \fi
  \defaultfontfeatures{Ligatures=TeX,Scale=MatchLowercase}
\fi
% use upquote if available, for straight quotes in verbatim environments
\IfFileExists{upquote.sty}{\usepackage{upquote}}{}
% use microtype if available
\IfFileExists{microtype.sty}{%
\usepackage{microtype}
\UseMicrotypeSet[protrusion]{basicmath} % disable protrusion for tt fonts
}{}
\usepackage[margin=1in]{geometry}
\usepackage{hyperref}
\hypersetup{unicode=true,
            pdftitle={Supplementary materials: Orienting the causal relationship between imprecisely measured traits using GWAS summary data},
            pdfborder={0 0 0},
            breaklinks=true}
\urlstyle{same}  % don't use monospace font for urls
\usepackage{graphicx,grffile}
\makeatletter
\def\maxwidth{\ifdim\Gin@nat@width>\linewidth\linewidth\else\Gin@nat@width\fi}
\def\maxheight{\ifdim\Gin@nat@height>\textheight\textheight\else\Gin@nat@height\fi}
\makeatother
% Scale images if necessary, so that they will not overflow the page
% margins by default, and it is still possible to overwrite the defaults
% using explicit options in \includegraphics[width, height, ...]{}
\setkeys{Gin}{width=\maxwidth,height=\maxheight,keepaspectratio}
\IfFileExists{parskip.sty}{%
\usepackage{parskip}
}{% else
\setlength{\parindent}{0pt}
\setlength{\parskip}{6pt plus 2pt minus 1pt}
}
\setlength{\emergencystretch}{3em}  % prevent overfull lines
\providecommand{\tightlist}{%
  \setlength{\itemsep}{0pt}\setlength{\parskip}{0pt}}
\setcounter{secnumdepth}{0}
% Redefines (sub)paragraphs to behave more like sections
\ifx\paragraph\undefined\else
\let\oldparagraph\paragraph
\renewcommand{\paragraph}[1]{\oldparagraph{#1}\mbox{}}
\fi
\ifx\subparagraph\undefined\else
\let\oldsubparagraph\subparagraph
\renewcommand{\subparagraph}[1]{\oldsubparagraph{#1}\mbox{}}
\fi

%%% Use protect on footnotes to avoid problems with footnotes in titles
\let\rmarkdownfootnote\footnote%
\def\footnote{\protect\rmarkdownfootnote}

%%% Change title format to be more compact
\usepackage{titling}

% Create subtitle command for use in maketitle
\newcommand{\subtitle}[1]{
  \posttitle{
    \begin{center}\large#1\end{center}
    }
}

\setlength{\droptitle}{-2em}
  \title{Supplementary materials: Orienting the causal relationship between
imprecisely measured traits using GWAS summary data}
  \pretitle{\vspace{\droptitle}\centering\huge}
  \posttitle{\par}
  \author{}
  \preauthor{}\postauthor{}
  \date{}
  \predate{}\postdate{}


\begin{document}
\maketitle

\subsection{S2 Text. Sensitivity analysis for measurement error on the
MR Steiger
test}\label{s2-text.-sensitivity-analysis-for-measurement-error-on-the-mr-steiger-test}

Assuming that either \(x \rightarrow y\) or \(y \rightarrow x\), the
causal direction can be inferred by evaluating which of \(\rho_{g, x}\)
and \(\rho_{g, y}\) is larger in magnitude. The Steiger test is a
hypothesis test that provides a p-value for observing the difference in
these correlations under the null hypothesis that they are equal.

Assuming the causal direction is \(x \to y\), two stage MR is formulated
using the following regression models:

\[
x = \alpha_1 + \beta_1 g + e_1
\]

for the first stage and

\[
y = \alpha_2 + \beta_2 \hat{x} + e_2
\]

where \(\hat{x} = \hat{\alpha}_1 + \hat{\beta}_1 g\). Writing in scale
free terms, \(\rho_{g, x}\) denotes the correlation between \(g\) and
the exposure variable \(x\), and it is expected that
\(\rho_{g, x} > \rho_{g, y}\) because
\(\rho_{g, y} = \rho_{g, x}\rho_{x, y}\), where \(\rho_{x, y}\) is the
causal association between \(x\) and \(y\) (which is likely to be less
than 1).

In the presence of measurement error in \(x\) and \(y\), however, the
empirical inference of the causal direction will instead be based on
evaluating \(\rho_{g, x_o} > \rho_{g, y_o}\), which can be simplified:

\[
\begin{aligned}
\rho_{g, x_O} & > \rho_{g, y_O} \\
\rho_{g, x} \rho_{x, x_O} & > \rho_{g,y}\rho_{y,y_O}\\
\rho_{g, x} \rho_{x, x_O} & > \rho_{g,x}\rho_{x,y}\rho_{y,y_O}\\
\rho_{x, x_O} & > \rho_{x,y}\rho_{y,y_O}
\end{aligned}
\]

In order to assess how reliable the inference of the causal direction is
in the presence of measurement imprecision, we can evaluate the range of
potential values of measurement error in \(x\) and \(y\) over which the
empirical difference in \(\rho_{g, x_o}\) and \(\rho_{g, y_o}\) would
return the wrong causal direction.

For different values of \(\rho_{x,x_o}\),
\(\rho_{g,x} = \frac{\rho_{g, x_o}}{\rho_{x,x_o}}\) and
\(\rho_{g,x_o} \leq \rho_{x,x_o} \leq 1\). For different values of
\(\rho_{y,y_o}\), \(\rho_{g,y} = \frac{\rho_{g, y_o}}{\rho_{y,y_o}}\)
and \(\rho_{g,y_o} \leq \rho_{y,x_o} \leq 1\).

Call \(z = \rho_{g,y} - \rho_{g,x}\) the true difference in the variance
explained by the genetic variant in \(y\) and \(x\). If \(z < 0\) then
we infer that \(x \rightarrow y\). There will be some values of
\(\rho_{x,x_o}\) and \(\rho_{y,y_o}\) that do not alter whether
\(z < 0\). To evaluate the reliability, \(R\), of the inference of the
causal direction with regards to measurement error, the objective is to
compare the proportion of the parameter space that agrees with the
inferred direction against the proportion which does not:

\[
R = \frac{V_{z \geq 0}}{ - V_{z < 0} }
\]

If \(R=1\) then the direction of causality is equally probable across
the range of possible measurement error values. If \(R > 1\) then \(R\)
times as much of the parameter space favours the inferred direction of
causality. \(V_{z}\), the total volume of the function (Figure 4), can
be obtained analytically by solving:

\[
\begin{aligned}
V_z & = \int^1_{\rho_{g,x_o}} \int^1_{\rho_{g,y_o}} \frac{\rho_{g,y_o}}{\rho_{y,y_o}} - \frac{\rho_{g,x_o}}{\rho_{x,x_o}}\,\,\,\, d\rho_{y,y_o}d\rho_{x,x_o} \\
& = \rho_{g,x_o}log(\rho_{g,x_o}) - \rho_{g,y_o}log(\rho_{g,y_o}) + \rho_{g,x_o}\rho_{g,y_o}(log(\rho_{g,y_o})-log(\rho_{g,x_o}))
\end{aligned}
\]

\(V_{z \ge 0}\), the proportion of the volume that lies above the
\(z=0\) plane, can also be obtained analytically. The region of this
volume is bound by the values of \(\rho_{x,x_o}\) and \(\rho_{y,y_o}\)
where \(0 = \rho_{g,y} - \rho_{g,x}\), which can be expanded to
\(\rho_{y,y_o} = \rho_{g,y_o}\rho_{x,x_o} / \rho_{g,x_o}\). Hence,

\[
\begin{aligned}
V_{z \ge 0} & = \int^1_{\rho_{g,x_o}} \int^{\frac{\rho_{g,y_o}\rho_{x,x_o}}{\rho_{g,x_o}}}_{\rho_{g,y_o}} \frac{\rho_{g,y_o}}{\rho_{y,y_o}} - \frac{\rho_{g,x_o}}{\rho_{x,x_o}}\,\,\,\, d\rho_{y,y_o}d\rho_{x,x_o} \\
& = 2\rho_{g,x_o}\rho_{g,y_o} - 2\rho_{g,y_o} - \rho_{g,y_o}log(\rho_{g,x_o}) - \rho_{g,x_o}\rho_{g,y_o}log(\rho_{g,x_o})
\end{aligned}
\]

Thus \(V_{z < 0} = V_{z} - V_{z \geq 0}\).


\end{document}
